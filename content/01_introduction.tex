\chapter{Introduction}
\label{cha:introduction}

\chapterquote{I'm awesome!}{Barney Stinson, WIRED magazine, 19.1.2009}



\begin{figure}[!ht]
	\centering
	\includegraphics[width=0.5\textwidth]{images/snet_logo_gray.png}\\
	\caption{Images can be loaded from the images folder and sub folders}
	\label{fig:introduction__loremipsum}
\end{figure}



\begin{description}
	\item[Lorem] ipsum
	\item[dolor] sit
	\item[amet] !
\end{description}



\section{Overview}
This thesis takes a look at the deployment methods and tools of Moodle. Moodle is a course management software. Two methods are compared, one is the classic deployment method using own resources and another is using cloud deployment methods and tools. The classic method and cloud deployment represent different ways of deploying of Moodle. When deploying with the classic method, hardware are needed which can be sold or rented from a company. Hosting provider can buy servers and installs the application on virtual machines. The another method is the hosting moodle on the amazon cloud. Different tools and technigues can be used when deploying on the cloud. 
\section{Problem statement}
\subsection{Deployment effort}

\subsection{Maintenance effort}
\section{Goal}
\section{Project scope}
\section{Structure of this thesis}



\begin{table}[!ht]
	\small
	\centering
	\begin{tabular}{|l|l|l|l|}
		\hline
		Lorem & ipsum & dolor & sit \\
		\hline
		amet & consetetur & sadipscing & elitr \\
		\hline
		Lorem & ipsum & dolor & sit \\
		\hline
		amet & consetetur & sadipscing & elitr \\
		\hline
	\end{tabular}
	\caption{Lorem ipsum...}
\end{table}




